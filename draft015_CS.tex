\documentclass[reqno,a4paper,11pt]{article}
\pdfoutput=1
\usepackage{xcolor}

%% \texorpdfstring{$L_\infty$}{Linfinity}

\usepackage{graphicx}
\usepackage[textwidth = 430 pt, textheight = 630 pt]{geometry}

\definecolor{MyDarkBlue}{rgb}{0.15,0.25,0.45}
\usepackage{epsfig,rotating}
\usepackage{amsmath,amssymb}
\usepackage{amsfonts}
\usepackage{mathrsfs}
\usepackage{bbm}
\usepackage[normalem]{ulem}
\usepackage{booktabs}
\usepackage{enumerate}

%\usepackage{bm}

\usepackage{booktabs}

\usepackage{latexsym}
\usepackage{amsthm}
\usepackage[all,knot]{xy}
\xyoption{arc}


%\usepackage[T1]{fontenc}
\usepackage[utf8x]{inputenc}

\usepackage{hyperref}
\hypersetup{
hypertexnames=false,
colorlinks=true,
citecolor=MyDarkBlue,
linkcolor=MyDarkBlue,
urlcolor=MyDarkBlue,
pdfauthor={Christian S\"amann and Lennart Schmidt},
pdftitle={Towards an M5-Brane Model II: Metric String Structures},
pdfsubject={hep-th math-ph},
breaklinks=true
}

\usepackage{tikz}
\usetikzlibrary{matrix,cd,arrows}
\usepackage{mathtools}
\usepackage[aligntableaux=center]{ytableau}
\usepackage[all,knot]{xy}
\xyoption{arc}

\newcommand{\triend}{\mbox{\hspace{0.2mm}}\hfill$\triangle$}
\newcommand{\black}{\mbox{\hspace{0.2mm}}\hfill$\blacksquare$}

%%%%%%%%%%%%%%%%%%%%%%%%%%%%%%%%%%%%%%%%%%%%%%%%%%%%%%%%%%%%%%%
%% Pseudo-jHEP/harvMac Anfang
%%%%%%%%%%%%%%%%%%%%%%%%%%%%%%%%%%%%%%%%%%%%%%%%%%%%%%%%%%%%%%%

\linespread{1.09}

\setlength{\footnotesep}{3.5mm}
\let\fn\footnote
\renewcommand{\footnote}[1]{\linespread{1.1}\fn{#1}\linespread{1.29}}

%\usepackage{fancyhdr}
%\pagestyle{fancy} \lhead{\jobname} \chead{} \rhead{\today} \lfoot{}
%\cfoot{\thepage} \rfoot{}
%\usepackage[left]{lineno}

\makeatletter\renewcommand{\section}{\@startsection
{section}{1}{\z@}{-3.5ex plus -1ex minus
    -.2ex}{2.3ex plus .2ex}{\bf }}
\makeatletter\renewcommand{\subsection}{\@startsection{subsection}{2}{\z@}{-3.25ex
plus -1ex minus
   -.2ex}{1.5ex plus .2ex}{\bf }}
\makeatletter\renewcommand{\subsubsection}{\@startsection{subsubsection}{3}{-2.45ex}{-3.25ex
plus -1ex minus -.2ex}{1.5ex plus .2ex}{\it }}
\renewcommand{\thesection}{\arabic{section}}
\renewcommand{\thesubsection}{\arabic{section}.\arabic{subsection}}
\renewcommand{\@seccntformat}[1]{\@nameuse{the#1}.~~}
\renewcommand{\thesubsubsection}{}
\renewcommand{\theequation}{\thesection.\arabic{equation}}
\providecommand*{\xhookrightfill@}{%
  \arrowfill@{\lhook\joinrel\relbar}\relbar\rightarrow
}
\providecommand*{\xhookrightarrow}[2][]{%
  \ext@arrow 0395\xhookrightfill@{#1}{#2}%
}
\makeatletter \@addtoreset{equation}{section}
\def\Ddots{\mathinner{\mkern1mu\raise\p@
\vbox{\kern7\p@\hbox{.}}\mkern2mu
\raise4\p@\hbox{.}\mkern2mu\raise7\p@\hbox{.}\mkern1mu}}
\setcounter{tocdepth}{2}

\usepackage[toc,page]{appendix}

\newtheorem{thm}{Theorem}[section]
\renewcommand{\thethm}{\thesection.\arabic{thm}}
\newtheorem{lemma}[thm]{Lemma}
\newtheorem{definition}[thm]{Definition}
\newtheorem{theorem}[thm]{Theorem}
\newtheorem{proposition}[thm]{Proposition}
\newtheorem{corollary}[thm]{Corollary}
\newtheorem{remark}[thm]{Remark}
\newtheorem{example}[thm]{Example}

\renewcommand{\appendices}{
\section*{Appendix}\label{appendices}\setcounter{subsection}{0}
\addcontentsline{toc}{section}{Appendix}
\setcounter{equation}{0}
\makeatletter
\renewcommand{\theequation}{\Alph{subsection}.\arabic{equation}}
\renewcommand{\thesubsection}{\Alph{subsection}}
\renewcommand{\thethm}{\Alph{subsection}.\arabic{thm}}
\@addtoreset{equation}{subsection}
\@addtoreset{thm}{subsection}
\makeatother
}



%%%%%%%%%%%%%%%%%%%%%%%%%%%%%%%%%%%%%%%%%%%%%%%%%%%%%%%%%%%%%%%
%% Pseudo-Harvmac Ende
%%%%%%%%%%%%%%%%%%%%%%%%%%%%%%%%%%%%%%%%%%%%%%%%%%%%%%%%%%%%%%%

%\hyphenation{mani-folds mani-fold opera-tor bet-ween}
%\usepackage{epsfig,rotating}
%\usepackage{amsmath,amssymb}
%\usepackage{amsfonts}
%\usepackage{mathrsfs}
%\usepackage{bbm}
%\usepackage{bm}

%\usepackage{graphicx}
%\usepackage{xypic}

\usepackage{macros} %% use the macros in macros.sty, edit as convenient


%% Makros only used in this paper
\newcommand{\clidf}{\Omega_{{\rm cl},\RZ}}
\newcommand{\Pair}{{\sf Pair}}
\newcommand{\hol}{{\sf hol}}
\newcommand{\dotsp}{\;\cdot\;}
\newcommand{\crit}{\mathrm{cr}}
\newcommand{\gstr}{\gamma_\mathrm{str}}
\newenvironment{conditions}{
\vspace{-2mm}\begin{itemize}
\setlength{\itemsep}{-1mm}
}{\vspace{-2mm}\end{itemize}}

\begin{document}
\begin{titlepage}
\begin{flushright}
 EMPG--??--??
\end{flushright}
\vskip2.0cm
\begin{center}
{\LARGE \bf Towards an M5-Brane Model II:\\[0.3cm] Metric String Structures}
\vskip1.5cm
{\Large Christian S\"amann and Lennart Schmidt}
\setcounter{footnote}{0}
\renewcommand{\thefootnote}{\arabic{thefootnote}}
\vskip1cm
{\em Maxwell Institute for Mathematical Sciences\\
Department of Mathematics, Heriot--Watt University\\
Colin Maclaurin Building, Riccarton, Edinburgh EH14 4AS, U.K.}\\[0.5cm]
{Email: {\ttfamily c.saemann@hw.ac.uk~,~ls27@hw.ac.uk}}
\end{center}
\vskip1.0cm
\begin{center}
{\bf Abstract}
\end{center}
\begin{quote}
We clarify the mathematical formulation of metric string structures, which play a crucial role in the formulation of six-dimensional superconformal field theories. We show that the connections on non-abelian gerbes usually introduced in the literature are problematic in that they are locally gauge equivalent to connections on an abelian gerbe. String structures form an exception and we introduce the general concept of an adjusted Weil algebra leading to potentially interacting connections on higher principal bundles. We then discuss the metric extension of string structures, the corresponding adjusted Weil algebra. The latter leads to connections that were previously constructed by hand in the context of gauged supergravities. We also explain how the Leibniz algebras constructed from an embedding tensor in gauged supergravities fit into our picture.
\end{quote}
\end{titlepage}

\tableofcontents

\section{Introduction and results}

\subsection{Overview}

Our understanding of M-theory would be vastly improved by a clean picture of the effective dynamics of stacks of multiplet M5-branes. These dynamics are governed by the so-called $(2,0)$-theory, a six-dimensional superconformal field theory, whose existence was postulated over 20 years ago~\cite{Witten:1995zh}. Attempts at constructing a classical Lagrangian of this theory have so far failed, and it is believed that such a Lagrangian does not exist, see e.g.~\cite{Witten:2007ct}. On closer inspection, however, many of the arguments against its existence are not conclusive~\cite{Saemann:2017zpd} and there may still be hope if we can identify the correct mathematical framework.

The $(2,0)$-theory involves a 2-form potential and deforming the free abelian theory to an interacting one is already a challenge. As proved in~\cite{Bekaert:9909094,Bekaert:2000qx}, there is no continuous such deformation. But this may be too much to ask; the Lagrangian may be of Chern--Simons type and therefore demand for a discrete coupling constant. This is the case in the M2-brane models and higher Chern--Simons terms indeed arise in the model presented in~\cite{Saemann:2017zpd}. 

Mathematically, the 2-form potential on a single M5-brane is a connection on a gerbe, a higher or categorified notion of an abelian principal bundle. It is therefore reasonable to turn towards connections on the non-abelian generalizations of gerbes introduced in the literature~\cite{Breen:math0106083,Aschieri:2003mw}. These are given in terms of local 1- and 2-forms and the additional 1-forms are required to circumvent the usual Eckmann--Hilton type argument that higher-dimensional parallel transport has to be abelian. 

At an abstract level, these allow for an elegant construction of 6d superconformal field equations via a higher-dimensional Penrose--Ward transform~\cite{Saemann:2012uq,Saemann:2013pca,Jurco:2014mva,Jurco:2016qwv}. Looking at concrete examples, however, suggests that the solutions of these equations are not particularly interesting. Similarly, direct constructions of a Lagrangian involving connections on non-abelian gerbes led to negative results, see e.g.~\cite{Ho:2012nt}.

As we show in section~\ref{ssec:fake_flat_trivial}, the connections defined in~\cite{Breen:math0106083,Aschieri:2003mw} are locally gauge-equivalent to abelian gerbes. While they are suitable for higher version of Chern--Simons theory, they necessarily fail in the description of field theories that may contain locally non-vanishing 2-form curvatures. This is, in fact, a rather general feature of connections on higher principal bundles. Higher gauge algebras are modeled by $L_\infty$-algebras, and each $L_\infty$-algebra comes with its own homotopy Maurer--Cartan theory, a generalisation of Chern--Simons theory. For every $L_\infty$-algebra, we thus obtain gauge potentials, curvatures, gauge transformations and Bianchi identities, that is, a full set of kinematical data for a (higher) gauge theory. The straightforward categorification of connections leads precisely to this type of kinematical data, which is suitable for higher Chern--Simons theories, but not for the purposes of non-topological higher gauge theories.

For certain $L_\infty$-algebras, however, there is a choice that one can make in the definition of kinematical data, which allows for connections on non-abelian gerbes which are {\em not} gauge equivalent to connections on abelian gerbes. One class of such $L_\infty$-algebras are the string Lie 2-algebras, higher analogues of the Lie algebras $\aspin(n)$. These are particularly interesting since their appearance in the description of the $(2,0)$-theory is expected for a number of reasons~\cite{Saemann:2019leg}. Furthermore, the string group $\sString(3)$ is a categorified version of $\sSpin(3)\cong \sSU(2)$~\cite{Saemann:2017rjm}, the simplest, interesting non-abelian Lie group.

In this paper, we derive in detail the local connection data, the appropriate notion of curvature, the gauge transformations as well as the Bianchi identities for two versions of the string Lie 2-algebra, allowing for an interpolation to general string Lie 2-algebra models. We also develop the metric extensions which are required for an action principle, and point out the relation to the resulting local connection data with the higher form curvatures obtained in the tensor hierarchies of gauged supergravities.

\subsection{The problem with non-abelian connections}

A definition of connections that allows for a straightforward generalization to $L_\infty$-algebras was given long ago by Henri Cartan~\cite{Cartan:1949aaa,Cartan:1949aab}, see~\cite{Sati:2008eg} for the complete generalization to $L_\infty$-algebras. In this approach, the dichotomy of Lie algebras and differential forms, the two basic ingredients in the local definition of connections, is overcome by switching from a Lie or $L_\infty$-algebra $\frg$ to its dual differential graded algebra (dga), known as the {\em Chevalley--Eilenberg algebra} $\sCE(\frg)$. Morphisms between $\sCE(\frg)$ and the dga of differential forms $(\Omega^\bullet(U),\dd)$ on some contractible patch $U$ on some manifold encode flat connections on $U$. Non-flat connections are obtained if one switches from $\frg$ to the corresponding $L_\infty$-algebra of inner derivations, $\inn(\frg)$, whose Chevalley--Eilenberg algebra is known as the {\em Weil algebra} $\sW(\frg)$ of $\frg$. To define global $\frg$-connection objects, one imposes constraints on the morphisms from $\sW(\frg)$ to $\Omega^\bullet(U)$. In particular, the morphism has to map a particular differential graded subalgebra of $\sW(\frg)$, the invariant polynomials $\inv(\frg)$ of $\frg$, to global objects. The invariant polynomials sit in the exact complex as follows:
\begin{equation}\label{eq:ses_1}
 \begin{tikzcd}
  \sCE(\frg) & \sW(\frg)\arrow[l] & \inv(\frg) \arrow[l]~~.
 \end{tikzcd} 
\end{equation}

The problem arising in the straightforward categorification of connections based on $L_\infty$-algebras is already visible at the level of the short exact complex~\eqref{eq:ses_1}. Recall that the appropriate notion of isomorphisms for $L_\infty$-algebra are given by quasi-isomorphisms and for the definitions of the Weil algebra and the invariant polynomials to be meaningful, one would clearly expect that a quasi-isomorphism $\phi:\frg\rightarrow \tilde \frg$ induces a chain of isomorphisms,
\begin{equation}\label{eq:ses_2}
 \begin{tikzcd}
  \sCE(\frg) \arrow[d,leftrightarrow]{}{\approxeq} & \sW(\frg)\arrow[l] \arrow[d,leftrightarrow]{}{\approxeq}& \inv(\frg) \arrow[l]\arrow[d,leftrightarrow]{}{\approxeq}\\
  \sCE(\tilde \frg) & \sW(\tilde \frg)\arrow[l] & \inv(\tilde \frg) \arrow[l]
 \end{tikzcd}
\end{equation}
While this is always true for Lie algebras, it fails to hold for general $L_\infty$-algebras. The conclusion is that the straightforward definition of the Weil algebra $\sW(\frg)$ as the Chevalley--Eilenberg algebra of the inner derivations $\inn(\frg)$ is problematic. 

Recall that in Cartan's approach, replacing the Chevalley--Eilenberg algebra with the Weil algebra allowed for non-flat connections, and in this formalism, the inconsistency of the Weil algebra with quasi-isomorphisms is corrected by forcing connections on non-abelian gerbes to be gauge equivalent to connections on abelian gerbes. Explicitly, one sees this in the gauge $L_\infty$-algebroid, which is commonly depicted dually in the form of the BRST complex. The BRST-transformations do not close, and the commutator of two gauge transformations induce additional transformations proportional to the so-called fake curvature forms. These are all $n$-form components of the total curvature of a higher connection except for the form component of highest degree. In the quantum field theory literature, such a BRST complex is commonly called {\em open}. It is then postulated that gauge transformations close only on-shell, requiring a lift to the BV complex, where the required equations of motion are imposed, cf.~the discussion in~\cite{Jurco:2018sby}. The problem for non-abelian connections, however, is that the equations of motion effectively render the connections abelian. As a consequence, it is very hard to write down gauge-invariant interacting actions.

The requirement that all fake curvature forms need to vanish has also been observed in the finite descriptions of connections in terms of parallel transport functors~\cite{Baez:2004in,Baez:0511710}. Here, it was found that a consistent parallel transport of strings is only invariant under surface reparametrizations if the fake curvature condition is met. As mentioned above, we show in section~\ref{ssec:fake_flat_trivial} that the corresponding connections are gauge equivalent to connections on abelian gerbes.

\subsection{String structures as an exception}

It turns out, however, that for particular $L_\infty$-algebras such as the string Lie 2-algebras, the definition of the Weil algebra can be modified~\cite{Sati:2008eg}, guaranteeing the expected compatibility of the exact complex~\eqref{eq:ses_1} with quasi-isomorphisms. The dga-morphisms to differential forms then yield the connections on higher generalizations of spin structures known as {\em string structures}~\cite{Killingback:1986rd,Witten:1987cg}, see section~\ref{ssec:local_connections_ordinary_string_structures}. They were, in fact, first discovered when trying to couple gauge potential to the Kalb--Ramond $B$-field in supergravity~\cite{Bergshoeff:1981um,Chapline:1982ww}. 

This picture, however, is incomplete for our purposes. First, string Lie 2-algebras can be modeled by various representatives in their quasi-isomorphism classes. The discussion in~\cite{Sati:2008eg} as well as the classical formulas of~\cite{Bergshoeff:1981um,Chapline:1982ww} are given only for {\em minimal} or {\em skeletal models}. For these, the underlying graded vector space underlying is minimal but the Jacobiator, which encodes the failure of the binary product to preserve the Jacobi identity, is non-trivial. As a consequence, these models are hard to integrate, cf.~\cite{Schommer-Pries:0911.2483}, and anything involving finite transformations such as the global involving corresponding principal 2-bundles becomes cumbersome. Also, it will be a consistency requirement for eventual higher gauge theories that they are agnostic about which representative of the quasi-isomorphism class of an $L_\infty$-algebra was used in their definition. We thus need to extend the discussion at least to the other extreme case of {\em strict models} of the string Lie-algebras, in which the Jacobiator is trivial. 

Second, the graded vector spaces underlying string Lie 2-algebra models are not symplectic and therefore do not admit a {\em cyclic structure}, the correct notion of an inner product for $L_\infty$-algebra. A solution to this problem is to use a procedure similar to introducing antifields in the BV formalism and to essentially double the relevant graded vector space by applying a degree-shifted cotangent functor~\cite{Saemann:2017zpd,Saemann:2017rjm}. This will lead to metric string structures which will be relevant to constructing action principles in the future and in particular to improving the model of~\cite{Saemann:2017zpd}. The full development of these metric string structures for general string Lie 2-algebra models is the main goal of this paper. As we point out in section~\ref{ssec:tensor_hierarchies}, these are also intimately related to the tensor hierarchies in supergravity and our formalism can make the construction of curvature forms much more systematic.

\subsection{Outline and results}

We begin in section~\ref{sec:L_infty_algebras} with a summary of $L_\infty$-algebras, and their associated differential graded algebras, the Chevalley-Eilenberg, Weil and free dgas. We recall the definition of quasi-isomorphisms\footnote{The compositions of $L_\infty$-algebra 2-morphisms and quasi-isomorphisms in the dga-picture are worked out in appendix~\ref{app:A:concatenation}.} in the dga-picture~\cite{Sati:2008eg} as well as the relevant structural theorems for $L_\infty$-algebras. The two models of the string Lie 2-algebra and their dgas as well as an extended description in terms of Lie 3-algebras are also introduced. 

In section~\ref{sec:inv_polys}, we recall the two evident notions of invariant polynomials of an $L_\infty$-algebra, $\inv(\frg)$ and $\overline{\inv}(\frg)$ from~\cite{Sati:2008eg} as well as the correspondence between Chevalley--Eilenberg cocycles and elements of $\overline{\inv}(\frg)$. We state our guiding principle that quasi-isomorphisms should preserve the exact complex~\eqref{eq:ses_1} in section~\ref{ssec:inv_and_quasi_isos}. Our first theorem then shows that compatibility of quasi-isomorphisms with~\eqref{eq:ses_1} induces the relevant compatibility with the second, more restrictive notion of invariant polynomials $\overline{\inv}(\frg)$. As an explicit example, we discuss the case of the Lie 2-algebra $\inn(\frh)$ of inner derivations of a general finite-dimensional Lie algebra $\frh$ and prove compatibility. In the remainder of the section, we present the necessary {\em adjustment} of the Weil algebra for both skeletal and strict models of the string Lie 2-algebra. We also show that the string Lie 2-algebra is special and that the way that the adjustment is done cannot be generalized to arbitrary Lie 2-algebras.

Section~\ref{sec:HGT} begins with an outline of Cartan's construction of connection forms and the general picture for $L_\infty$-algebras given in~\cite{Sati:2008eg}. We then present a simple generalization of the AKSZ formalism~\cite{Alexandrov:1995kv}, which allows for a rapid construction of the gauge $L_\infty$-algebroid for local connection data in the form of the BRST complex. As stated above, one can easily read off the required consistency conditions of a higher gauge theory from this complex. In section~\ref{ssec:unadjusted_lead_to_fake_flatness}, we show that the usual concept of Weil algebra leads to the problematic fake curvature conditions and introduce the concept of {\em adjusted Weil algebra}, for which this is not the case. To underline the importance of adjusted Weil algebras, we prove that for principal 2-bundles, fake flatness reduces the kinematical data to that of an abelian gerbe in section~\ref{ssec:fake_flat_trivial}.

Section~\ref{sec:metric_string_structures} is then devoted to the main result: the connection data on string structures and their metric extensions. Explicit results for the non-metric case are presented in section~\ref{ssec:local_connections_ordinary_string_structures}, while the metric case is given in section~\ref{ssec:local_connections_metric_string_structures}.

We point out applications and give an outlook of planned future research in section~\ref{sec:outlook}, beginning with the evident application of self-dual 3-forms in six-dimensions, BPS self-dual strings in four dimensions and the supersymmetric extensions of these. We then turn to the tensor hierarchies in gauged supergravities~\cite{deWit:2008ta}. Here, the embedding tensor induces a Leibniz algebra, which comes with an associated Lie 2-algebra. This has been shown recently in~\cite{Sheng:2015:1-5,Kotov:2018vcz}; we merely add that this follows rather immediately from the general theory of weak Lie 2-algebras~\cite{Roytenberg:0712.3461}. As an example, we give the Leibniz algebra that yields the string Lie 2-algebra under this correspondence. We close with a few remarks on why we need to go beyond local string structures in order to potentially construct M5-brane models in the future.

We note that some of the results presented this paper were previously published in contributions~\cite{Schmidt:2019pks} to the LMS/EPSRC Durham Symposiums ``Higher Structures in M-Theory.''

\section{\texorpdfstring{$L_\infty$}{L-infinity}-algebras and associated differential graded algebras}\label{ssec:CE-algebras}

In this section, we review basic algebraic structures important for the construction of higher gauge theory. We introduce $L_\infty$-algebras and their Chevalley--Eilenberg description in terms of a differential graded commutative\footnote{All our differential graded algebras will be commutative, and we shall drop this adjective from here on almost everywhere.} algebra (dga). We also introduce the associated Weil and free differential graded algebras. Finally, we review how quasi-isomorphisms, which form the relevant type of isomorphisms for $L_\infty$-algebras, are described in this picture.

\subsection{Chevalley--Eilenberg algebra of an \texorpdfstring{$L_\infty$}{L-infinity}-algebra}\label{sec:L_infty_algebras}

A natural and convenient categorification of the notion of a Lie algebra is given by what are called strong homotopy Lie algebras, or $L_\infty$-algebras for short. 

\begin{definition}
 An \underline{$L_\infty$-algebra} $\frg$ consists of a $\RZ$-graded vector space $\frg = \bigoplus_{k\in \RZ} \frg_k$ together with a set of totally antisymmetric, multilinear maps \mbox{$\mu_i :  \wedge^i \frg \to \frg,~i\in \NN,$} of degree~$2-i$, which satisfy the \underline{higher} or \underline{homotopy Jacobi identities}
\begin{equation}\label{eq:hom_rel}
 \sum_{i+j=n}\sum_{\sigma\in S_{i|j}}\chi(\sigma;a_1,\ldots,a_{i})(-1)^{j}\mu_{j+1}(\mu_i(a_{\sigma(1)},\ldots,a_{\sigma(i)}),a_{\sigma(i+1)},\ldots,a_{\sigma(n)})=0
\end{equation}
for all $n\in\NN^+$ and $a_1,\dots,a_n\in \frg$, where the second sum runs over all $(i,j)$-unshuffles $\sigma\in S_{i|j}$. An \underline{$n$-term $L_\infty$-algebra}, or \underline{Lie $n$-algebra}\footnote{There is a slight distinction, but we shall use both terms synonymously in this paper.}, is an $L_\infty$-algebra that is concentrated (i.e.~non-trivial only) in degrees $-n+1,\dots,0$. The \underline{trivial $L_\infty$-algebra} is the $L_\infty$-algebra $\frg=\bigoplus_{k\in\RZ}\frg_k$ with $\frg_k=\{0\}$.
\end{definition}
Here, an {\em unshuffle} $\sigma\in S_{i|j}$ is a permutation whose image consists of ordered tuples $\big(\sigma(1),\dots,\sigma(i)\big)$ and $\big(\sigma(i+1),\dots,\sigma(i+j)\big)$. Moreover, $\chi(\sigma; a_1,\dots,a_n)$ denotes the {\em graded antisymmetric Koszul sign} defined by the graded antisymmetrized products
\begin{equation}
a_1 \dots   a_n = \chi(\sigma;a_1,\dots,a_n)a_{\sigma(1)} \dots  a_{\sigma(n)}~,
\end{equation}
where any transposition not involving an even element acquires a minus sign. 

As shown in \cite{Baez:2003aa}, a 2-term $L_\infty$-algebra is equivalent to a (semi-strict) Lie 2-algebra, that is, a categorification of a Lie algebra, which relaxes the Jacobi identity up to a natural isomorphism. This is evident in the lowest few homotopy Jacobi relations, i.e.
\begin{equation}\label{eq:hJacobi_Lie_2}
\begin{aligned}
0&=\mu_1\left(\mu_1\left(a_1\right)\right)~,\\
0&= \mu_1\left(\mu_2\left(a_1,a_2\right)\right)-
\mu_2\left(\mu_1\left(a_1\right),a_2\right)+
(-1)^{\left| a_1\right|  \left| a_2\right| } \mu_2\left(\mu_1\left(a_2\right),a_1\right)~,\\
0 &=\mu_1\left(\mu_3\left(a_1,a_2,a_3\right)\right)-\mu_2\left(\mu_2\left(a_1,a_2\right),a_3\right)+(-1)^{|a_2| |a_3|} \mu_2(\mu_2(a_1,a_3),a_2)+\\
&~~~~+(-1)^{|a_1| |a_3|} \mu_2(\mu_2(a_2,a_3),a_1)+
\mu_3(\mu_1(a_1),a_2,a_3)-
(-1)^{|a_1| |a_2|} \mu_3(\mu_1(a_2),a_1,a_3)-\\
&~~~~-
(-1)^{|a_1| |a_3|} \mu_3(\mu_1(a_3),a_1,a_2)~,
\end{aligned}
\end{equation}
where $a_i \in \frg$. These relations state that $\mu_1$ is a graded differential compatible with $\mu_2$, and $\mu_2$ is a generalization of a Lie bracket with the violation of the Jacobi identity controlled by $\mu_3$.

There is an alternative and elegant way of describing an $L_\infty$-algebra $\frg$ and its higher Jacobi relations in terms of a coalgebra and coderivations \cite{Lada:1992wc}, cf.~also~\cite{Lada:1994mn}. To see this, consider the grade-shifted vector space $\frg[1]$, where the square bracket refers to a degree shift of all elements of $\frg$ by $-1$ and, correspondingly, of all coordinate functions by $+1$, cf.~\cite{Jurco:2018sby}. This degree shift induces a shift of the degree of the maps $\mu_i$ from $2-i$ to $1$ and allows to define a degree~$1$ coderivation
\begin{equation}
\CD:\odot^\bullet \frg[1]\to\odot^\bullet \frg[1]~,
\end{equation}
which acts on the graded symmetric coalgebra $\odot^\bullet \frg[1]$ generated by $\frg[1]$. 

More explicitly, $\odot^\bullet \frg[1]$ is spanned by graded symmetric elements $a_1 \odot\dots\odot\, a_n$ and is equipped with the coproduct 
\begin{equation}
\Delta(a_1\odot\dots\odot a_n) = \sum\limits_{i+j=n}\,\,\sum\limits_{\sigma\in S_{i|j}} \epsilon(\sigma;a_1,\dots,a_n) (a_{\sigma(1)}\odot\dots\odot a_{\sigma(i)})\otimes(a_{\sigma(i+1)}\odot\dots\odot a_{\sigma(n)})~,
\end{equation}
where $S_{i|j}$ again denotes the set of $(i,j)$-unshuffles and $\epsilon$ is now the {\em graded symmetric Koszul sign}, which is related to the graded antisymmetric Koszul sign via
\begin{equation}
\epsilon(\sigma;a_1,\ldots,a_n) = \text{sgn}(\sigma)\chi(\sigma;a_1,\dots,a_n)~.
\end{equation}
A {\em coderivation} $\CD$ is now given by a linear map $\CD:\odot^\bullet \frg[1]\to\odot^\bullet\frg[1]$ which satisfies the co-Leibniz rule
\begin{equation}
\Delta\circ \CD = (\CD \otimes \text{id}+\text{id} \otimes \CD) \circ \Delta~.
\end{equation}
We note that the higher products $\mu_i$ induce maps from $\odot^i \frg[1]\rightarrow \frg[1]$, which can be extended to codifferentials $\CD_i$. The sum of all these codifferentials combine into a new codifferential $\CD$, encoding the $L_\infty$-algebra $\frg$.

The third way of describing $L_\infty$-algebras, which is the important one for this paper, is now the dualization of the above coalgebra description. In the case of ordinary Lie algebras, this yields what is known as the Chevalley--Eilenberg algebra of a Lie algebra.

\begin{definition}
 The \underline{Chevalley--Eilenberg} algebra $\sCE(\frg)$ of an $L_\infty$-algebra $\frg$ encoded in a codifferential $\CD$ is the differential graded commutative algebra
 \begin{equation}
  \sCE(\frg)~=~\big(\,\CC^\infty(\frg[1])\cong \wedge^\bullet \frg[1]^*\,,\,Q\,\big)~,
 \end{equation}
 where $Q=\CD^*$ is the \underline{homological vector field}, i.e.~a vector field on $\frg[1]$ of degree~1 satisfying $Q^2=0$, which acts as a differential on $\CC^\infty(\frg[1])$. We call the graded vector space $\frg[1]$ the \underline{differential graded (dg)-manifold}\footnote{Recall that a {\em differential graded manifold} is a manifold with a differential on the algebra of smooth functions. These are often called $Q$-manifolds in the literature due to the homological vector field $Q$ inducing the differential. An important example of a dg-manifold is the grade-shifted tangent bundle $T[1]M$ of a manifold $M$, where the algebra of smooth functions is identified with the differential forms on $M$ and $Q$ is identified with the de Rham differential.} corresponding to the $L_\infty$-algebra $\frg$.
\end{definition}
\noindent Note that we are only interested in $L_\infty$-algebras whose graded vector spaces are simple enough (e.g.~finite dimensional) to allow for a linear dual. 

As a first  example, consider a finite-dimensional Lie algebra $\frg$. Then $\frg[1]$ comes with coordinate functions $t^\alpha$ with respect to some basis and the homological vector field is of the form $Q=-\tfrac12 f^\alpha_{\beta\gamma} t^\beta t^\gamma$. The condition $Q^2=0$ is equivalent to the $f^\alpha_{\beta\gamma}$ being the structure constants of a Lie algebra.

As a second example, consider a Lie 2-algebra (or 2-term $L_\infty$-algebra) $\frg=\frg_{-1}\oplus \frg_0$. Let $(t^\alpha,r^a)$ be the generators of $\frg[1]^*$ with degrees~1 and~2. The homological vector fields $Q$ acts on the generators of $\sCE(\frg)$ according to
\begin{equation}\label{eq:CE_Lie_2_algebra}
 Q~:~t^\alpha\mapsto -f^\alpha_a r^a-\tfrac12 f^\alpha_{\beta\gamma}t^\beta t^\gamma~,~~~r^a\mapsto-f^a_{\alpha b}t^\alpha r^b-\tfrac{1}{3!}f^a_{\alpha\beta\gamma}t^\alpha t^\beta t^\gamma~,
\end{equation}
where the structure constants $f^\alpha_a, f^\alpha_{\beta\gamma}, f^a_{\alpha b}, f^a_{\alpha\beta\gamma}\in \FR$ satisfy relations corresponding to~\eqref{eq:hJacobi_Lie_2} or, equivalently, to $Q^2=0$.

An immediate advantage of the dga-perspective on $L_\infty$-algebras is that the appropriate notion of morphism is immediately clear:
\begin{definition}
A \underline{morphism of $L_\infty$-algebras} $\phi:\frg\rightarrow \tilde \frg$ is (the dual of) a morphism of differential graded algebras between the corresponding Chevalley--Eilenberg algebras $\sCE(\frg)$ and $\sCE(\tilde \frg)$, 
\begin{equation}
 \Phi:\big(\CC^\infty(\frg[1]),Q\big)\rightarrow \big(\CC^\infty(\tilde \frg[1]),\tilde Q\big)~.
\end{equation}
In particular, $\Phi$ is of degree~$0$ and respects the differential: $\Phi \circ Q = \tilde Q \circ \Phi$. If $\Phi$ is invertible, we call $\phi$ an \uline{isomorphism of $L_\infty$-algebras}.
\end{definition}

In the dual picture, this translates to a collection of totally antisymmetric, multilinear maps $\phi_i:\wedge^i\frg\to\tilde \frg$ of degree~$1-i$ satisfying
\begin{subequations}\label{eq:L_infty_morphism}
\begin{equation}
\begin{aligned}
   &\sum_{j+k=i}\sum_{\sigma\in S(j|i)}~(-1)^{k}\chi(\sigma;a_1,\ldots,a_i)\phi_{k+1}(\mu_j(a_{\sigma(1)},\dots,a_{\sigma(j)}),a_{\sigma(j+1)},\dots ,a_{\sigma(i)})\\
   \ &=\ \sum_{j=1}^i\frac{1}{j!} \sum_{k_1+\cdots+k_j=i}\sum_{\sigma\in{\rm Sh}(k_1,\ldots,k_{j-1};i)}\chi(\sigma;a_1,\ldots,a_i)\zeta(\sigma;a_1,\ldots,a_i)\,\times\\
   &\kern1cm\times \mu'_j\Big(\phi_{k_1}\big(a_{\sigma(1)},\ldots,a_{\sigma(k_1)}\big),\ldots,\phi_{k_j}\big(a_{\sigma(k_1+\cdots+k_{j-1}+1)},\ldots,a_{\sigma(i)}\big)\Big)
\end{aligned}
\end{equation}
with the sign $\zeta(\sigma;a_1,\ldots,a_i)$ given by
\begin{equation}\label{eq:zeta-sign}
 \zeta(\sigma;a_1,\ldots,a_i)\ :=\ (-1)^{\sum_{1\leq m<n\leq j}k_mk_n+\sum_{m=1}^{j-1}k_m(j-m)+\sum_{m=2}^j(1-k_m)\sum_{k=1}^{k_1+\cdots+k_{m-1}}|a_{\sigma(k)}|_\frg}~.
\end{equation}
\end{subequations}

For example, a morphism $\phi$ of 2-term $L_\infty$-algebras $\phi:\frg\rightarrow \tilde \frg$ consists of maps $\phi_1:\frg\rightarrow \tilde \frg$ and $\phi_2:\frg \wedge \frg\rightarrow \tilde \frg$ of degrees $0$ and $-1$, respectively. The higher products on $\frg$ and $\tilde \frg$ are then related by the following formulas:
\begin{equation}\label{eq:Lie_2_algebra_morph}
\begin{aligned}
0 &= \phi_1(\mu_1(w_1)) - \mu'_1(\phi_1(w_1))~,\\
0 &= \phi_1(\mu_2(w_1,w_2)) - \mu'_1(\phi_2(w_1,w_2))-\mu'_2(\phi_1(w_1),\phi_1(w_2))~,\\
0 &= \phi_1(\mu_2(w_1,v_1)) +\phi_2(\mu_1(v_1),w_1) - \mu'_2(\phi_1(w_1),\phi_1(v_1))~,\\
0 &= \phi_1(\mu_3(w_1,w_2,w_3)) -\phi_2(\mu_2(w_1,w_2),w_3) + \phi_2(\mu_2(w_1,w_3),w_2)\\
&\phantom{{}={}} - \phi_2(\mu_2(w_2,w_3),w_1) - \mu'_3(\phi_1(w_1),\phi_1(w_2),\phi_1(w_3)) \\
&\phantom{{}={}} + \mu'_2(\phi_1(w_1),\phi_2(w_2,w_3))- \mu'_2(\phi_1(w_2),\phi_2(w_1,w_3))\\
&\phantom{{}={}}+\mu'_2(\phi_1(w_3),\phi_2(w_1,w_2))~,
\end{aligned}
\end{equation}
where $w_i$ and $v_i$ denote elements of $\frg$ of degrees $0$ and $-1$, respectively.

We note that a morphism of $L_\infty$-algebras is invertible and thus an isomorphism of $L_\infty$-algebras if and only if $\phi_1$ is invertible. This is very clear in the above explicit formulas~\eqref{eq:Lie_2_algebra_morph} for a Lie 2-algebra morphism. Note that an $L_\infty$-algebra isomorphism preserves the dimensions of the graded subspaces $\frg_k$ of its source $L_\infty$-algebra $\frg=\oplus_{k\in \RZ}\frg$. In most cases, this notion of isomorphism is too restrictive, and we shall return to this point in section~\ref{ssec:quasi_isos}.

For more details on the three descriptions of $L_\infty$-algebras in terms of higher products, differential graded coalgebras and differential graded algebras, see e.g.~\cite[Appendix A]{Jurco:2018sby}.

Finally, we note that the description of $L_\infty$-algebroids in terms of a dg-algebra immediately allows for 

\subsection{Weil algebra and free algebra}\label{ssec:Weil_and_free}

Given an $L_\infty$-algebra $\frg$, it is natural to consider the corresponding $L_\infty$-algebra of inner derivations, as done e.g.~in~\cite{Sati:2008eg}. Its Chevalley--Eilenberg is known as the Weil algebra of $\frg$ and it will play a major role in our discussion.

\begin{definition}[\cite{MR0042426},\cite{Sati:2008eg}] 
 The \underline{Weil algebra} of an $L_\infty$-algebra $\frg$ is the graded commutative algebra
 \begin{subequations}
 \begin{equation}
  \sW(\frg)=\odot^\bullet(\frg[1]^*\oplus \frg[2]^*)~,
 \end{equation}
 which becomes a differential graded commutative algebra with the differential $Q_\sW$ defined by
 \begin{equation}
  Q_\sW|_{\frg[1]^*}=Q_\sCE+\sigma\eand Q_\sW|_{\frg[2]^*}=-\sigma Q_\sCE \sigma^{-1}~,
 \end{equation}
 \end{subequations}
 where $Q_\sCE$ is the Chevalley--Eilenberg differential on $\frg[1]^*$ and $\sigma:\frg[1]^*\rightarrow \frg[2]^*$ is the shift isomorphism of degree~1. Note that indeed $Q_\sW^2=0$. We denote the dual $L_\infty$-algebra by $\inn(\frg)$, that is $\sW(\frg)=\sCE(\inn(\frg))$.
\end{definition}

The natural embedding $i:\frg\embd \inn(\frg)$ is an $L_\infty$-algebra morphism, as one readily checks. Its dual yields the projection
\begin{equation}
 i^*:\sW(\frg)\twoheadrightarrow\sCE(\frg)~,
\end{equation}
which is a morphism of dgas, in particular $Q_{\sCE} i^*=i^*Q_{\sW}$. The kernel of $i^*$ is the ideal in $\sW(\frg)$ generated by $\frg[2]^*$. Moreover, we have an isomorphism $\sCE(\frg)\cong \sW(\frg)/\ker(i^*)$.

It is now useful to introduce the subalgebra 
\begin{equation}
 \sW_{\rm h}(\frg):=\odot^\bullet\frg[2]^*
\end{equation}
of {\em horizontal} elements in the Weil algebra. Note that $Q_\sW$ does not necessarily close on $\sW_{\rm h}(\frg)$ and that $\sW_{\rm h}$ is in the kernel of $i^*$.

As examples, we construct the Weil algebras of a generic Lie (1-)algebra and a generic Lie 2-algebra. Let $\frg$ be an ordinary, finite-dimensional Lie algebra $\frg$ and let $t^\alpha\in \frg[1]^*$, $\alpha=1,\dots,d$, be coordinate functions on $\frg[1]$, which are of degree~$1$. We also introduce the coordinate functions $\hat t^\alpha=\sigma t^\alpha\in\frg[2]^*$ on $\frg[2]$, which are of degree~$2$. The Weil algebra $\sW(\frg)$ is then the polynomial algebra generated by $t^\alpha$ and $\hat t^\alpha$, and the Weil differential acts as
\begin{equation}\label{eq:Weil_of_ordinary_Lie}
Q_\sW~:~t^\alpha \mapsto -\tfrac12 f^\alpha_{\beta\gamma} t^\beta t^\gamma + \hat t^\alpha \eand
\hat t^\alpha\mapsto -f^\alpha_{\beta\gamma} t^\beta \hat t^\gamma~,
\end{equation}
where $f^\alpha_{\beta\gamma}$ are again the structure constants of $\frg$. 

For the case of a Lie 2-algebra $\frg=(\frg_{-1}\rightarrow \frg_0)$, recall the generators and the form of the Chevalley--Eilenberg algebra $\sCE(\frg)$ from~\eqref{eq:CE_Lie_2_algebra}. We introduce additional shifted generators $\hat t^\alpha=\sigma t^\alpha$ and $\hat r^a=\sigma r^a$, and the Weil differential acts as
\begin{equation}\label{eq:Weil_algebra_generic_Lie_2}
 \begin{aligned}
   Q_\sW~&:~&t^\alpha &\mapsto -\tfrac12 f^\alpha_{\beta\gamma} t^\beta t^\gamma-f^\alpha_a r^a+\hat t^\alpha~,\\
   &&\hat t^\alpha&\mapsto -f^\alpha_{\beta\gamma}t^\beta \hat t^\gamma+f^\alpha_a \hat r^a~,\\
   &&r^a&\mapsto-\tfrac1{3!}f^a_{\alpha\beta\gamma}t^\alpha t^\beta t^\gamma-f^a_{\alpha b}t^\alpha r^b+\hat r^a~,\\
   &&\hat r^a&\mapsto \tfrac12 f^a_{\alpha\beta\gamma}t^\alpha t^\beta \hat t^\gamma+f^a_{\alpha b}\hat t^\alpha r^b-f^a_{\alpha b}t^\alpha \hat r^b
 \end{aligned}
\end{equation}
with the same structure constants $f^\alpha_a, f^\alpha_{\beta\gamma},f^a_{\alpha b}, f^a_{\alpha\beta\gamma}\in \FR$ as appearing in~\eqref{eq:CE_Lie_2_algebra}. The relation $Q_\sW^2=0$ follows by construction.

Note that a morphism between the Chevalley--Eilenberg algebras of two $L_\infty$-algebras $\frg$ and $\tilde \frg$ readily lifts to a morphism between their Weil algebras. In particular, $\Phi:\sCE(\frg)\to\sCE(\tilde \frg)$, mapping generators $a\in \sCE(\frg)$ to $\Phi(a)$, can be lifted to a morphism $\hat{\Phi}:\sW(\frg)\to\sW(\tilde \frg)$ using  $\sigma a \mapsto \sigma \Phi(a)$, because the following diagrams commute:
\begin{equation}\label{eq:lift_CE_map_to_W}
\begin{tikzcd}
a \arrow[d,"\hat\Phi"]\arrow[r,"Q_\sW"] & Q_\sCE a + \sigma a\arrow[d,"\hat\Phi"] & \sigma a\arrow[d,"\hat\Phi"]\arrow[r,"Q_\sW"] & -\sigma Q_\sCE a\arrow[d,"\hat\Phi"] \\
\Phi(a)\arrow[r,"Q_\sW"] & Q_{\sCE} \Phi(a) + \sigma\Phi(a) &\sigma \Phi(a)\arrow[r,"Q_{\sW}"] &-\sigma Q_{\sCE} \Phi(a)
\end{tikzcd}
\end{equation}
for $a$ a generator of $\sCE(\frg)$. 

Closely related to the Weil algebra is the {\em free algebra}\footnote{These are also called {\em free differential algebra} in the supergravity literature, see~\cite{Castellani:1995gz} and references therein.} $\sF(\frg)$ of an $L_\infty$-algebra $\frg$, which is given by
\begin{equation}
 \sF(\frg)\coloneqq ( \odot^\bullet(\frg[1]^*\oplus\frg[2]^*),Q_\sF = \sigma)~,
\end{equation}
where $\sigma:\frg[1]^*\to\frg[2]^*$ is again the shift isomorphism. In fact, the Weil algebra $\sW(\frg)$ is naturally isomorphic to the corresponding free algebra $\sF(\frg)$, because we have the morphisms
\begin{equation}
\begin{aligned}
\Upsilon: \sF(\frg)\to\sW(\frg),\quad &a\mapsto a~,~~~~~&\Upsilon^{-1}:\sW(\frg)\to\sF(\frg),\quad & a \mapsto a~,\\
& \hat a \mapsto Q_\sW a~,~~~~~&&\hat a \mapsto \hat a - Q_\sCE a~,
\end{aligned}
\end{equation}
where $a\in \frg[1]^*$ and $\hat a:=\sigma a\in\frg[2]^*$, with $\Upsilon^{-1}\circ \Upsilon=\id_{\sF(\frg)}$ and $\Upsilon\circ \Upsilon^{-1}=\id_{\sW(\frg)}$. Note that these maps are indeed dga-morphisms, because the following diagrams commute: 
\begin{equation}\label{eq:weil_free_iso}
\begin{tikzcd}
a \arrow[r,"Q_\sF"]\arrow[d,"\Upsilon"] & \hat a\arrow[d,"\Upsilon"] & \hat a \arrow[d,"\Upsilon"]\arrow[r,"Q_\sF"] & 0 \arrow[d,"\Upsilon"] \\
a \arrow[r,"Q_\sW"] & Q_\sW a & Q_\sW a\arrow[r,"Q_\sW"] & 0
\end{tikzcd}
\end{equation}
and
\begin{equation}
\begin{tikzcd}
a \arrow[r,"Q_\sW"]\arrow[d,"\Upsilon^{-1}"] & Q_\sW a \arrow[d,"\Upsilon^{-1}"] & \hat a \arrow[r,"Q_\sW"]\arrow[d,"\Upsilon^{-1}"] & -\sigma Q_\sCE a\arrow[d,"\Upsilon^{-1}"] \\
a \arrow[r,"Q_\sF"] & Q_\sCE a + \hat a - Q_\sCE a & \hat a -Q_\sCE a\arrow[r,"Q_\sF"] & -\sigma Q_\sCE a
\end{tikzcd}
\end{equation}

\subsection{Quasi-isomorphisms and 2-morphisms}\label{ssec:quasi_isos}

As indicated above, it turns out that in most cases, the appropriate notion of isomorphism of $L_\infty$-algebras is {\em not} a bijective $L_\infty$-algebra morphism, but a generalization known as a  quasi-isomorphism. In the higher product picture, we readily extend the corresponding definition from cochain complexes:
\begin{definition}
 An $L_\infty$-algebra \uline{quasi-isomorphism} $\phi:\frg\rightarrow \frh$ is a morphism of $L_\infty$-algebras, $\phi:\frg\rightarrow \frh$, which induces an isomorphism on cohomology\footnote{Recall that $\phi_1$ is a chain map and therefore descends to cohomology.},
 \begin{equation}
  \phi_1: H^\bullet_{\mu_1}(\frg)\xrightarrow{~\cong~}H^\bullet_{\mu_1}(\frh)~.
 \end{equation}
 We write $\frg\approxeq\frh$ for quasi-isomorphic $L_\infty$-algebras $\frg$ and $\frh$.
\end{definition}
\noindent It is clear that quasi-isomorphisms form an equivalence relation. In particular, they are transitive by definition: morphisms of $L_\infty$-algebras $\phi:\frg \rightarrow \frh$ and $\psi:\frh \rightarrow \frl$ can be composed to a morphism $\psi\circ \phi: \frg\rightarrow \frl$, which descends to the composition of the isomorphisms on the cohomologies.

The definition of a quasi-isomorphisms can be reformulated as categorical equivalence, see e.g.~\cite{Baez:2003aa} for the example of 2-term $L_\infty$-algebras, and this picture is readily translated to the dga description of $L_\infty$-algebras:
\begin{proposition}[\cite{Sati:2008eg}]
 A quasi-isomorphism between $L_\infty$-algebras $\frg$ and $\frh$ is equivalent to a pair of dga-morphisms 
\begin{equation}
    \begin{tikzcd}
    \sCE(\frg) \arrow[r,bend left=30,"\Phi"{name=D}] &\sCE(\frh)\arrow[l,bend left=30,"\Psi"{name=U,below}]
    \end{tikzcd}
  \ewith \Psi\circ \Phi\cong \id_{\sCE(\frg)}\eand \Phi\circ \Psi\cong \id_{\sCE(\frh)}~.
\end{equation}
We call the triple $(\Phi,\Psi,\eta)$ a \underline{dual quasi-isomorphism} and say that $\sCE(\frg)$ and $\sCE(\frh)$ are \underline{dually quasi-isomorphic}.\footnote{Note that this nomenclature is important as a quasi-isomorphism of the dgas $\sCE(\frg)$ and $\sCE(\frh)$ is not the same as the dual of a quasi-isomorphism between $\frg$ and $\frh$ since the former refers to $Q$-cohomologies, while the latter refers to $\mu_1$-cohomologies.} 
\end{proposition}

For this proposition to be meaningful, we clearly need a notion of 2-morphisms of dga-algebras. In the case of differential graded vector spaces, 2-morphisms are simply chain homotopies, but respecting the algebra product makes the definition now slightly more involved. It is helpful to note that 2-morphisms between morphisms from free dgas into arbitrary dgas are again obvious to define. These can then be extended to 2-morphisms between arbitrary dgas~\cite{Sati:2008eg}.


Given $L_\infty$-algebras $\frg$ and $\frh$, together with dga-morphisms $\Phi:\sCE(\frg)\rightarrow \sCE(\frh)$ and $\Psi:\sCE(\frg)\rightarrow \sCE(\frh)$, a 2-morphism $\eta$ between $\Phi$ and $\Psi$,
 \begin{equation}\label{eq:2-morph_1}
    \begin{tikzcd}
    \sCE(\frh) &\sCE(\frg)\arrow[l,bend left=50,"\Psi"{name=U,below}]\arrow[l,bend right=50,"\Phi"{name=D},swap]
    \arrow[Rightarrow,"\eta", from=D, to=U,start anchor={[yshift=-1ex]},end anchor={[yshift=1ex]}]
    \end{tikzcd}~,
 \end{equation}
can then be extended to a 2-morphism between morphisms from $\sF(\frg)$ to $\sCE(\frh)$ as follows
\begin{equation}
\begin{tikzcd}
&\sCE(\frg)\arrow[dl,bend right=30,"\Phi",swap]&&\frg[2]^*\\
\sCE(\frh) &&\sW(\frg)\arrow[ur,hookleftarrow]\arrow[ul,"i^*",swap,""{name=U,below}]\arrow[dl,"i^*"]&\sF(\frg)\arrow[l,swap,"\Upsilon"]\\
&\sCE(\frg)\arrow[ul,bend left=30,"\Psi"{name=D}]&& \arrow[Rightarrow,"\eta",from=U,to=D,start anchor={[xshift=-1ex]},end anchor={[yshift=1ex,xshift=1ex]}]
\end{tikzcd}
\end{equation}
where $\eta$ should vanish on $\frg[2]^*\hookrightarrow\sW(\frg)$ in order to undo the extension from $\sCE(\frg)$ to $\sW(\frg)$. For convenience, let us also introduce the following pullbacks to $\sF$:
\begin{equation}
 \Phi_\sF:=\Phi\circ i^*\circ \Upsilon\eand \Psi_\sF:=\Psi\circ i^*\circ \Upsilon~.
\end{equation}
These considerations lead to the following definition.

\begin{definition}[\cite{Sati:2008eg}]\label{def:2-morphism}
 A \underline{2-morphism} $\eta$ from $\Phi$ to $\Psi$ as in~\eqref{eq:2-morph_1} is given by a linear map $\eta$ of degree~$-1$ on the generators of the free algebra $\sF(\frg)$,
 \begin{subequations}
    \begin{equation}
    \eta: \frg[1]^*\oplus\frg[2]^*\to \sCE(\frh)~,
    \end{equation}
which is continued to all of $\sF(\frg)$ by the formula
 \begin{equation}
\begin{aligned}\label{eq:expansion_formula}
&\eta: a_1 \dots   a_n\mapsto\tfrac{1}{n!} \sum\limits_{\sigma \in S_n} \eps(\sigma)(a_1,\dots,a_n) \times\\
&\hspace{3cm}
\sum\limits_{k=1}^n (-1)^{\sum\limits_{i=1}^{k-1}\left|a_{\sigma(i)}\right|} \Phi_\sF(a_{\sigma(1)}  \dots  a_{\sigma(k-1)}) \eta(a_{\sigma(k)})  \Psi_\sF(a_{\sigma(k+1)} \dots  a_{\sigma(n)})~,
\end{aligned}
\end{equation}
 \end{subequations}
for $a_i \in \frg[1]^*\oplus\frg[2]^*$ to a chain homotopy on $\sF(\frg)$, 
 \begin{equation}\label{eq:2_morph_cond}
  \Phi_\sF-\Psi_\sF:=\Phi\circ i^*\circ \Upsilon-\Psi\circ i^*\circ \Upsilon=[Q,\eta]=Q_{\sCE}\circ\eta+\eta\circ Q_\sF~,
 \end{equation}
 and which becomes trivial when restricted to the generators $\Upsilon^{-1}(\frg[2]^*)$ of $\sW_{\rm h}(\frg)$. Here, $\eps(\sigma;a_1,\dots,a_n)$ is the symmetric\footnote{as opposed to the antisymmetric Koszul sign $\chi(\sigma;a_1,\dots,a_n)$ introduced earlier} Koszul sign of the permutation $\sigma$ of $a_1,\dots,a_n$. 
\end{definition}

A few remarks on this definition are in order. First, we note that it suffices to ensure condition~\eqref{eq:2_morph_cond} on the generators of $\sF(\frg)$ as the continuation~\eqref{eq:expansion_formula} then extends this property to all of $\sF(\frg)$. Second, the triviality upon restriction to $\Upsilon^{-1}(\frg[2]^*)$ means that $\eta: \sF(\frg)\to \sCE(\frh)$ induces a map $\eta_\sW:\sW(\frg)\to \sCE(\frh)$ which can be defined by its image of the generators $a\in \frg[1]^*$. The fact that $\eta_\sW$ vanishes on all $\sigma a \in \sW(\frg)$ then fixes its image of $Q_\sW a$ inside $\sW(\frg)$ and on $\sigma_\sF a$ insides $\sF(\frg)$. In particular, we have
\begin{equation}\label{eq:restiction_eta}
 \eta(\sigma_\sF a)=\eta(Q_\sCE a)
\end{equation}
on generators $a\in \sF(\frg)$. Third, for $\eta_\sW=\eta\circ \Upsilon^{-1}$ we have
\begin{equation}
(\Phi_\sW-\Psi_\sW)(a):=(\Phi\circ i^*-\Psi\circ i^*)(a)=(Q_\sCE\circ\eta_\sW+\eta_\sW\circ Q_\sW)(a)
\end{equation}
on the generators $a$ of $\sW(\frg)$ since $\Upsilon$ is a dga-isomorphism. Very importantly, however, this formula does {\em not} extend to all of $\sW(\frg)$: since $\Upsilon^{-1}(\hat a)$ for $a\in\frg[1]^*$ is not necessarily a homogeneous polynomial in the generators, the continuation formula~\eqref{eq:expansion_formula} does {\em not} have a simple analogue on $\sW(\frg)$. Fourth, let us stress that definition~\ref{def:2-morphism} naturally extends to 2-morphisms between morphisms between Weil algebras as $\sW(\frg)$ can certainly be seen as the Chevalley--Eilenberg algebra $\sCE(\inn(\frg))$. Fifth, 2-morphisms can be composed horizontally and vertically, and details are presented in appendix~\ref{app:A:concatenation}, where also the composition of quasi-isomorphisms is discussed. Finally, we have the following lemma, which directly follows from~\eqref{eq:expansion_formula}:
\begin{lemma}\label{lem:eta_as_derivation}
 If $\eta$ is a 2-morphism from $\Phi$ to $\Psi$ as in~\eqref{eq:2-morph_1} with $\Phi(-)=0$, then $\eta$ acts on $\sF(\frg)$ as a derivation.
\end{lemma}

It is instructive to spell out what this definition means for the example of morphisms between Lie 2-algebras $\frg=\frg_{-1}\oplus \frg_0$ and $\tilde \frg=\tilde \frg_{-1}\oplus \tilde \frg_0$. Recall our choice of generators $(t^\alpha,r^a)$ and the action of the Chevalley--Eilenberg differential $Q$ encoded in the structure constants $f^\alpha_a, f^\alpha_{\beta\gamma}, \dots$ from~\eqref{eq:CE_Lie_2_algebra}. We introduce analogous generators $(\tilde t^\mu,\tilde r^m)$ and a differential $\tilde Q$ encoded in structure constants $\tilde f^\mu_m, \tilde f^\mu_{\nu\kappa}, \dots$ for $\tilde \frg$. The morphisms $\Phi$ and $\Psi$ are defined by their images of the generators of $\frg[1]^*$:
\begin{equation}
 \Phi~:~t^\alpha\mapsto \Phi^\alpha_\mu t^\mu~,~~~r^a\mapsto\Phi^a_m\tilde r^m+\tfrac12 \Phi^a_{\mu\nu}\tilde t^\mu\tilde t^\nu~.
\end{equation}

To fix the 2-morphism, we note that a generic map $\eta:\frg[1]^*\oplus \frg[2]^*\rightarrow \sCE(\tilde \frg)$ of degree~$-1$ has the images
\begin{equation}
\eta~:~t^\alpha\mapsto 0~,~~~
r^a\mapsto \eta^a_\mu \tilde t^\mu~,
\end{equation}
which implies that the map $\eta_\sW$, taking generators of $\sW(\frg)$ to $\sCE(\tilde \frg)$, satisfies
\begin{equation}
\eta_\sW (t^\alpha) = 0~,~~~
\eta_\sW (r^a) = \eta^a_\mu \tilde t^\mu~.
\end{equation}
The requirement that $\eta_\sW$ vanishes along $\frg[2]^*\subset\sW(\frg)$ together with the formula \eqref{eq:expansion_formula} then also defines $\eta_\sW$ on $Q_\sW t^\alpha$ and $Q_\sW r^a$, which we use to calculate
\begin{equation}
\begin{aligned}
[Q,\eta] t^\alpha &= \tilde Q_\sCE(\eta_\sW(t^\alpha))-\eta_\sW(Q_\sW t^\alpha)= f^\alpha_a\eta^a_\mu \, \tilde t^{\mu}~,\\
[Q,\eta] r^a &= \tilde Q_\sCE(\eta_\sW(r^a))-\eta_\sW( Q_\sW r^a)\\
&=\eta^a_\mu(-f^\mu_m\tilde r^m-\tfrac12 f^\mu_{\nu\kappa}\tilde t^\nu\tilde t^\kappa)+\tfrac12 f^a_{\alpha b}(\Psi^\alpha_\mu \eta^b_\nu+\eta^b_\mu\Psi^\alpha_\nu)\tilde t^\mu \tilde t^\nu~.
\end{aligned}
\end{equation}
The condition $\Psi_\sW-\Phi_\sW=[Q,\eta_\sW]$ then translates to
\begin{equation}
\begin{aligned}
\Phi^\alpha_\mu-\Psi^\alpha_\mu &= f^\alpha_a \eta^a_\mu~,\\
\Phi^a_m - \Psi^a_m &= - \eta^a_\mu f^\mu_m~,\\
\Phi^a_{\left[\mu\nu\right]}-\Psi^a_{\left[\mu\nu\right]} &= -\eta^a_\kappa f^\kappa_{\mu\nu}+f^a_{\alpha b} (\Psi^\alpha_\mu\eta^b_\nu+\eta_\mu^b\Psi^\alpha_\nu)~,
\end{aligned}
\end{equation}
and this agrees with the familiar condition for 2-morphisms as given in~\cite{Baez:2009:aa}, cf.\ also appendix~A of~\cite{Sati:2008eg}.

As an example of a quasi-isomorphism, let us show that the Weil algebra $\sW(\frg)$ of an $L_\infty$-algebra $\frg$ is quasi-isomorphic to the Weil algebra $\sW(*)$ of the trivial $L_\infty$-algebra. We have already shown that the Weil algebra $\sW(\frg)$ is quasi-isomorphic to the free algebra $\sF(\frg)$, so it merely remains to show that $\sF(\frg)\approxeq\sW(*)$. The relevant morphisms are obvious,
\begin{equation}
 \begin{tikzcd}
    \sF(\frg)  \arrow[r,bend left=30]{}{\Phi} & \sW(\ast)  \arrow[l,bend left=30]{}{\Psi}
 \end{tikzcd}~,
\end{equation}
with 
\begin{equation}
\Phi(-)=0 \eand \Psi:0 \mapsto 0~.
\end{equation}
Clearly, $\Phi\circ \Psi=\id_0$, so it remains to find a 2-morphism $\eta:\Psi\circ\Phi\rightarrow \id_{\sF(\frg)}=0$. There is only one generic choice, namely 
\begin{equation}
 \eta(a)=\begin{cases}
       \sigma^{-1}(a) &\mbox{ for } a\in \im(\sigma)~,\\
       0 &\mbox{ else}~.
      \end{cases}
\end{equation}
for generators $a$ of $\sF(\frg)$, where $\sigma$ is the shift isomorphism in $\sF(\frg)$. We then have $[Q,\eta]_{\sF(\frg)}=\id_{\sF(\frg)}=\Psi\circ \Phi-\id_{\sF(\frg)}$.

The map $\eta$ can now be used to show that the $Q$-cohomology of $\sF(\frg)$ is trivial: given an $\alpha\in \sF(\frg)$ with $Q\alpha=0$, we have $\alpha=\id_{\sF(\frg)}(\alpha)=[Q,\eta] (\alpha)=Q(\eta(\alpha))$ and therefore any $Q$-closed algebra element is $Q$-exact.

The isomorphism $\Upsilon$ between $\sF(\frg)$ and $\sW(\frg)$ allows us to translate this argument to $\sW(\frg)$:
\begin{lemma}\label{lem:Q_cohomology_of_Weil_algebra_trivial}
 The $Q$-cohomology of the Weil algebra $\sW(\frg)$ of an $L_\infty$-algebra $\frg$ is trivial.
\end{lemma}
\noindent To prove this lemma, consider an $\alpha\in \sW(\frg)$ with $Q\alpha=0$. We then have $\beta=\Upsilon^{-1}(\alpha)\in \sF(\frg)$ which is exact and closed, $\beta=Q_\sF (\eta(\beta))$. It follows that $Q_\sW\Upsilon(\eta(\beta))=\Upsilon(Q_\sF(\eta(\beta)))=\Upsilon(\beta)=\alpha$ and $\alpha$ is thus exact.


\subsection{Important structural theorems for $L_\infty$-algebras}

Let us briefly recall some important structural theorems for $L_\infty$-algebras, which will simplify our discussion.
\begin{definition}
 Let $\frg$ be an $L_\infty$-algebra with higher products $\mu_i$, $i\in \NN^+$. We call $\frg$
 \begin{itemize}
  \setlength{\itemsep}{-1mm}
  \item[$\triangleright$] \uline{strict} if $\mu_i=0$ for $i\geq 3$ and $\frg$ is thus simply a differential graded Lie algebra;
  \item[$\triangleright$] \uline{minimal} if $\mu_1=0$;
  \item[$\triangleright$] \uline{linearly contractible} if $\mu_i=0$ for $i>1$ and $H^\bullet_{\mu_1}(\frg)=0$.
 \end{itemize}
\end{definition}
Fundamentally, we have the following theorem.
\begin{theorem}[Decomposition theorem, cf.~\cite{Kajiura:0306332}]
 Any $L_\infty$-algebra is isomorphic as an $L_\infty$-algebra to the direct sum of a minimal and a linearly contractible $L_\infty$-algebra.
\end{theorem}
Applying a projection to the minimal part of an $L_\infty$-algebra (which evidently induces an isomorphism on the $\mu_1$-cohomology), we immediately arrive at the following theorem, which historically predates the above one:
\begin{theorem}[cf.~\cite{kadeishvili1982algebraic,Kajiura:0306332}]
 Any $L_\infty$-algebra is quasi-isomorphic to a minimal $L_\infty$-algebra.
\end{theorem}
We can thus endow the cohomology $H^\bullet_{\mu_1}(\frg)$ of an $L_\infty$-algebra $\frg$ with an $L_\infty$-algebra structure such that it is quasi-isomorphic to $\frg$ itself. The resulting $L_\infty$-algebra is minimal in the sense that it is a dimensionally smallest representative of the quasi-isomorphism class of $\frg$. It is therefore called a {\em minimal model} of $\frg$.

Finally, we have another extreme example, relating $L_\infty$-algebras to differential graded Lie algebras:
\begin{theorem}[\cite{igor1995,Berger:0512576}]
 Any $L_\infty$-algebra is quasi-isomorphic to a strict $L_\infty$-algebra.
\end{theorem}

Let us discuss the example of a Lie 2-algebra $\frg=\frg_{-1}\oplus \frg_0$ in more detail. First, we note that there is an exact sequence
\begin{equation}\label{eq:es_Lie2}
 0\longrightarrow\ker(\mu_1) \hooklongrightarrow \frg_{-1}\xrightarrow{~\mu_1~} \frg_0 \xrightarrow{~\pi~} {\rm coker}(\mu_1)\longrightarrow 0~,
\end{equation}
where ${\rm coker}(\mu_1)$ carries a Lie algebra structure induced by $\mu_2$. A minimal model of $\frg$ has underlying graded vector space
\begin{equation}
 \frg^\circ=\frg^\circ_{-1}\oplus \frg^\circ_0\cong\ker(\mu_1)\oplus {\rm coker}(\mu_1)~.
\end{equation}
Using the decomposition theorem, we can further decompose $\frg$ (non-canonically) according to
\begin{equation}\label{eq:LAsplitting}
 \frg~=~\frg_{-1}\oplus \frg_0~=\left(
\begin{tikzcd}[row sep=2pt,column sep=1.7cm]
\frg_{-1}^0=\ker(\mu_1) & \frg^0_0\cong {\rm coker}(\mu_1)  \\
\oplus & \oplus\\
\frg_{-1}^1\cong \im(\mu_1) \arrow[r,"\mu_1=\text{id}"] & \frg^1_0=\im(\mu_1)
\end{tikzcd}
 \right)
\end{equation}
with the only non-trivial higher products being 
\begin{equation}
 \mu_1:\frg^1_{-1}\rightarrow \frg^1_0~,~~~\mu_2:\frg^0_0\wedge \frg^0_0\rightarrow \frg^0_0~,~~~\mu_2:\frg^0_0\wedge \frg^0_{-1}\rightarrow \frg^0_{-1}~,~~~\mu_3:\wedge^3 \frg^0_0\rightarrow \frg^0_{-1}~.
\end{equation}
In particular, $\frg^0_0$ is a Lie algebra, $\frg^0_{-1}$ is a $\frg^0_0$-module with action induced by $\mu_2$ and $\mu_3$ is an element of the Lie algebra cohomology group $H^3(\frg^0_0,\frg^0_{-1})$.


\subsection{String Lie 2-algebra models}

In the vast category of $L_\infty$-algebras, there are particularly interesting objects which are obtained by extending metric Lie algebras by cocycles which are in transgression to particular invariant polynomials. As we shall explain later, they allow for interesting constructions of higher gauge theories.

For the application to string theory, the simplest non-trivial one, the string Lie 2-algebra, plays a particular role. In the following, we discuss the relevant algebraic structures, their metric extensions, as well as two useful Lie 2-algebra models.

The string group $\sString(n)$ sits in the sequence
\begin{equation}\label{eq:whitehead_tower}
\begin{tikzcd}[column sep=20pt]
\dots \arrow[r] & \sString(n)\arrow[r] & \sSpin(n)\arrow[r] &\sSpin(n)\arrow[r] & \sSO(n)\arrow[r] & \sO(n)~,
\end{tikzcd}
\end{equation}
which is known as the {\em Whitehead tower} of $\sO(n)$. It is constructed by successively removing the lowest homotopy group: $\pi_0(\sO(n))$ is removed in the step from $\sO(n)$ to $\sSO(n)$, $\pi_1(\sO(n))$ in the step to $\sSpin(n)$ and $\pi_2(\sO(n))$ is already trivial. The string group $\sString(n)$ is obtained by removing $\pi_3(\sO(n))$. That is, $\sString(n)$ is a 3-connected cover of $\sSpin(n)$~\cite{Stolz:1996:785-800}. This definition only determines $\sString(n)$ up to homotopical equivalence and, as such, there are a variety of models.

Particularly interesting models are given by Lie 2-groups, and one example is that of~\cite{Schommer-Pries:0911.2483}. As shown in~\cite{Demessie:2016ieh}, this Lie 2-group can be differentiated to a minimal $L_\infty$-algebra which we call, following the categorical nomenclature, the skeletal model. This Lie 2-algebra allows for an immediate generalization to arbitrary metric Lie algebras, cf. also~\cite{Baez:2003aa}:
\begin{definition}
 Let $\frg$ be a Lie algebra endowed with a metric $(-,-)$. The \uline{skeletal model of the string Lie 2-algebra} or, simply, the \uline{skeletal string algebra} is the 2-term $L_\infty$-algebra 
 \begin{subequations}
 \begin{equation}
\astringsk(\frg) = \big(\,\,\FR[1] \overset{0}{\longrightarrow} \frg\,\,\big)
\end{equation}
with non-trivial brackets
\begin{equation}\label{eq:skel_string_algebra_brackets}
\begin{aligned}
 \mu_2&:\frg \wedge \frg\rightarrow \frg~,~~~&\mu_2(a_1,a_2)&=[a_1,a_2]~,\\
 \mu_3&:\frg \wedge \frg \wedge \frg\rightarrow \FR~,~~~&\mu_3(a_1,a_2,a_3)&=(a_1,[a_2,a_3])~,
\end{aligned}
\end{equation}
where $[-,-]$ is the commutator in $\frg$.
\end{subequations}
\end{definition}

This is an example of

The Weil algebra $\sW(\astringsk(\frg))$ is generated by coordinate functions $t^\alpha, r$ of degrees~1 and~2, respectively, together with their shifted copies $\hat t^\alpha=\sigma t^\alpha$ and $\hat r=\sigma r$ of degrees~2 and~3. The differential corresponding to \eqref{eq:skel_string_algebra_brackets} is then

Also, see \ref{hello} and \cite{whatnot}

The the repetition is other other interesting. 

Principle bundle principle bundle principal bundle

a M5-brane a M2-brane an M5-brane

This is teh theory and the theory...

Pseudo-Riemannian Yang-Mills

Yang-Mills

I'm loosing my mind

This is not worth while.

A self-Dual field conuration

And dealing with m-branes or d-branes

See \ref{44} and cf.~\ref{52}

Here are a few problems e.g. this one i.e. this one cf. also this one.

These ones are ok a few problems e.g.~this one i.e.~this one cf.~also this one.

Colloquialisms:

We'll isn't doesn't won't Don't contractions kind of sort of so-called This necessitates nothing.

Shouldn' trigger: JHEP23

This is an inline equation $5+2=3$with no space.

This is an inline equation $5+2=3$-test is ok

This is an inline equation$5+2=3$ with no space.

This is an inline equation$5+2=3$ with no space .
This is ok {$L_\infty$}-test.

This is a - test of , and . and ! and ? other signs.

This is a order -1 effect \footnote{Space too much}

and forms are of degree -2 and grade $+1$ order 5.

\subsection{all is not well}

\subsection{All is not well...}

A $\CN=4$ invariant action is given by a $m$-dimensional a $d$-dimensional.

A algebra a one-dimensional vector a unusual picture

$i=i+1$\footnote{No footnote attached to formula.}

This needs space 3 space

an Abelian group

\begin{equation}
Testing .
\end{equation}
although lower case continuation

\begin{equation}
\begin{aligned}
Testing .
\end{aligned}
\end{equation}
although lower case continuation

\begin{equation}
\begin{gathered}
Testing ,
\end{gathered}
\end{equation}
Although uower case continuation

\begin{equation}
\begin{gathered}
Testing 
\end{gathered}
\end{equation}
Although uower case continuation

\begin{equation}
\begin{gathered}
Testing 
\end{gathered}
\end{equation}
where this comma is missing

Indeed, this 22 text is indeed not readily read as is readily written.

And a paragraph

not ending in a fullstop!

\bibliography{bigone}

\bibliographystyle{latexeu}

\end{document}


